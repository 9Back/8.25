% TEMPLATE for Usenix papers, specifically to meet requirements of
%  USENIX '05
% originally a template for producing IEEE-format articles using LaTeX.
%   written by Matthew Ward, CS Department, Worcester Polytechnic Institute.
% adapted by David Beazley for his excellent SWIG paper in Proceedings,
%   Tcl 96
% turned into a smartass generic template by De Clarke, with thanks to
%   both the above pioneers
% use at your own risk.  Complaints to /dev/null.
% make it two column with no page numbering, default is 10 point

% Munged by Fred Douglis <douglis@research.att.com> 10/97 to separate
% the .sty file from the LaTeX source template, so that people can
% more easily include the .sty file into an existing document.  Also
% changed to more closely follow the style guidelines as represented
% by the Word sample file. 

% Note that since 2010, USENIX does not require endnotes. If you want
% foot of page notes, don't include the endnotes package in the 
% usepackage command, below.

% This version uses the latex2e styles, not the very ancient 2.09 stuff.

% Updated July 2018: Text block size changed from 6.5" to 7"

\documentclass[letterpaper,twocolumn,10pt]{article}
\usepackage{usenix2019,epsfig,endnotes}
\begin{document}

%don't want date printed
\date{}

%make title bold and 14 pt font (Latex default is non-bold, 16 pt)
\title{\Large \bf 9Back : Making Our Plans Great Again}

%for single author (just remove % characters)
\author{
{\rm Maxwell Bland}
\and
{\rm Leon Cheung}\\ \\
University of California, San Diego
\and
{\rm Kilian Pfeiffer}\\
% copy the following lines to add more authors
% \and
% {\rm Name}\\
%Name Institution
} % end author

\maketitle

% Use the following at camera-ready time to suppress page numbers.
% Comment it out when you first submit the paper for review.
\thispagestyle{empty}


\subsection*{Abstract}
The following paper describes a set of benchmarks run on the Plan 9 operating system. In particular, it uses the only actively developed branch, 9FRONT "RUN FROM ZONE!" (2018.09.09.0 Release). The goal of this project is to provide researchers and enthusiasts with greater insight into the current state of the operating system's performance and capabilities of interaction with hardware. Through tests of CPU, Memory, Network, and File System operations, we will gain insight on bottlenecks in the system's performance and the interactions between low-level (hardware) and high-level (OS) system components. These performance statistics will be contrasted with subjective experiences of ``responsiveness''.

\section{Introduction}

Plan 9 from Bell labs is a distributed operating system which emerged around the 1980's. It built upon the ideas of UNIX, but adopted an ideology that ``everything is a file''. Although the system was marketed in the 90's, it did not catch on, as prior operating systems had already gained enough of a foothold. Eventually, during the 2000's, Bell Labs ceased development on Plan 9, meaning official development halted. Unofficial\footnote{This is debatable. If you adopt an orphan, are they not your official child?} development continues on the 9front fork of the codebase, with new support for Wi-Fi, Audio, and everything anyone could ever want or need. 

The experiments were performed as a group using a shared codebase and a single machine described in the following section. The measurements were performed via programs written in Plan 9's \textit{special} version of C 99 and Alef, both described in the original Plan 9 paper \todo{cite}. The compilers used were the x86 versions of Plan 9's C compiler and Alef compiler, 8c and 8al respectively. The compilers were run with no special optimization settings. Version numbers are not available. Measurements were performed on a single machine running Plan 9 directly from hardware; given the nature of the Plan 9 system, additional metrics could be established for networked file systems and CPU servers; these measurements were not done for sake of simplicity, and because results under these conditions should be inferrable from the results cataloged within this paper.

\section{Machine Description}

We ran this beautiful operating system of the gods on a Thinkpad T420, the machine of true developers.

{\tt \small
\begin{verbatim}
    Processor: model, cycle time, cache sizes (L1, L2,
      instruction, data, etc.)
      Intel(R) Core(TM) i5-2520M CPU @ 2.50GHz 
      cache size 3072 KB
      cpu family 6
      model 42
      stepping 7
      siblings 4
      cores 2
      fpu yes
      fpu_exception yes
      fpu_exception yes
      bogomips 4986.98
      clflush size 64
      cache_alignment 64

    Memory bus
      DDR3-1333
      i/o-bus-frequency: 666MHz
      bus-bandwith:  10656 MB/s
      memory-clock:  166MHz
      Column Access Strobe (CAS) latency:

    I/O bus
      SataIII-speed: 600MB/s

    RAM size
      8 GB
    Disk: capacity, RPM, controller cache size
      Samsung SSD 860 EVO 500GB
      Capacity: 500GiB
      RPM: 550MB/s read, 520 MB/s write
       and 98,000 IOPS (Read QD32)
      Controller Cache Size: 512MB 
    Network card speed:
     Intel 82579 LM Gigabyte: 1Gb/s
     intel Centrino Ultimate-N 6300: 450 Mbps
      
    Operating system (including version/release) 
      9FRONT "RUN FROM ZONE!" (2018.09.09.0 Release)
\end{verbatim}
}

\section{This Section has SubSections}
\subsection{First SubSection}

Here's a typical figure reference.  The figure is centered at the
top of the column.  It's scaled.  It's explicitly placed.  You'll
have to tweak the numbers to get what you want.\\

% you can also use the wonderful epsfig package...
\begin{figure}[t]
\begin{center}
\begin{picture}(300,150)(0,200)
\put(-15,-30){\special{psfile = fig1.ps hscale = 50 vscale = 50}}
\end{picture}\\
\end{center}
\caption{Wonderful Flowchart}
\end{figure}

This text came after the figure, so we'll casually refer to Figure 1
as we go on our merry way.

\subsection{New Subsection}

It can get tricky typesetting Tcl and C code in LaTeX because they share
a lot of mystical feelings about certain magic characters.  You
will have to do a lot of escaping to typeset curly braces and percent
signs, for example, like this:
``The {\tt \%module} directive
sets the name of the initialization function.  This is optional, but is
recommended if building a Tcl 7.5 module.
Everything inside the {\tt \%\{, \%\}}
block is copied directly into the output. allowing the inclusion of
header files and additional C code." \\

Sometimes you want to really call attention to a piece of text.  You
can center it in the column like this:
\begin{center}
{\tt \_1008e614\_Vector\_p}
\end{center}
and people will really notice it.\\

\noindent
The noindent at the start of this paragraph makes it clear that it's
a continuation of the preceding text, not a new para in its own right.


Now this is an ingenious way to get a forced space.
{\tt Real~$*$} and {\tt double~$*$} are equivalent. 

Now here is another way to call attention to a line of code, but instead
of centering it, we noindent and bold it.\\

\noindent
{\bf \tt size\_t : fread ptr size nobj stream } \\

And here we have made an indented para like a definition tag (dt)
in HTML.  You don't need a surrounding list macro pair.
\begin{itemize}
\item[]  {\tt fread} reads from {\tt stream} into the array {\tt ptr} at
most {\tt nobj} objects of size {\tt size}.   {\tt fread} returns
the number of objects read. 
\end{itemize}
This concludes the definitions tag.

\subsection{How to Build Your Paper}

You have to run {\tt latex} once to prepare your references for
munging.  Then run {\tt bibtex} to build your bibliography metadata.
Then run {\tt latex} twice to ensure all references have been resolved.
If your source file is called {\tt usenixTemplate.tex} and your {\tt
  bibtex} file is called {\tt usenixTemplate.bib}, here's what you do:
{\tt \small
\begin{verbatim}
latex usenixTemplate
bibtex usenixTemplate
latex usenixTemplate
latex usenixTemplate
\end{verbatim}
}


\subsection{Last SubSection}

Well, it's getting boring isn't it.  This is the last subsection
before we wrap it up.

\section{Acknowledgments}

A polite author always includes acknowledgments.  Thank everyone,
especially those who funded the work. 

\section{Availability}

It's great when this section says that MyWonderfulApp is free software, 
available via anonymous FTP from

\begin{center}
{\tt ftp.site.dom/pub/myname/Wonderful}\\
\end{center}

Also, it's even greater when you can write that information is also 
available on the Wonderful homepage at 

\begin{center}
{\tt http://www.site.dom/\~{}myname/SWIG}
\end{center}

Now we get serious and fill in those references.  Remember you will
have to run latex twice on the document in order to resolve those
cite tags you met earlier.  This is where they get resolved.
We've preserved some real ones in addition to the template-speak.
After the bibliography you are DONE.

{\normalsize \bibliographystyle{acm}
\bibliography{../common/bibliography}}


\theendnotes

\end{document}







